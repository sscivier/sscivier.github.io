\documentclass[11pt,a4paper]{article}
\usepackage[utf8]{inputenc}
\usepackage[T1]{fontenc}
\usepackage{lmodern}
\usepackage[margin=0.75in]{geometry}
\usepackage{titlesec}
\usepackage{enumitem}
\usepackage{hyperref}
\usepackage{xcolor}
\usepackage{fontawesome}
\usepackage{microtype}
\usepackage{etaremune}

% Color scheme
\definecolor{darkblue}{RGB}{25,25,112}
\definecolor{linkcolor}{RGB}{14,165,233}

% Hyperref setup
\hypersetup{
    colorlinks=true,
    linkcolor=linkcolor,
    urlcolor=linkcolor,
    citecolor=linkcolor,
    pdfauthor={Sam A. Scivier},
    pdftitle={Sam A. Scivier - Curriculum Vitae},
    pdfsubject={Curriculum Vitae},
    pdfkeywords={geophysics, machine learning, uncertainty quantification, earthquake prediction}
}

% Section formatting 
\titleformat{\section}
{\large\bfseries}
{}{0pt}{\MakeUppercase}

\titleformat{\subsection}
{\normalsize\bfseries}
{}{0pt}{}

% Professional summary formatting - narrower margins and slightly smaller font
\newenvironment{summary}
  {\begin{quote}\small}
  {\end{quote}}

% Compact list formatting
\setlist[itemize]{leftmargin=1.5em, topsep=0pt, itemsep=2pt}
\setlist[enumerate]{leftmargin=1.5em, topsep=0pt, itemsep=2pt}

% Remove page numbers
\pagestyle{empty}

% Compact entry command
\newcommand{\cventry}[4]{%
    \noindent\textbf{#1} \hfill #2\\
    \textit{#3} \hfill #4\\[0.3em]
}
% Education entry command
% Usage: \eduentry{degree}{subject}{institution}{location}{date}{expected date (opt)}{funding (opt)}{supervisors (opt)}{description (opt)}
\newcommand{\eduentry}[9]{%
    \par\noindent\hspace{1.5em}%
    \begin{minipage}[t]{\dimexpr\textwidth-1.5em\relax}
    \textbf{\textit{#1}}, \textbf{#2} \hfill #5\\
    \textbf{#3}, \textit{#4} \hfill #6\\
    \if\relax\detokenize{#7}\relax\else {\small Funding: #7}\\\fi
    \if\relax\detokenize{#8}\relax\else {\small Advisor(s): #8}\\\fi
    \if\relax\detokenize{#9}\relax\else{\small #9}\\\fi
    \end{minipage}\par
    \vspace{0.5em}
}

% Research entry command
% Usage: \resentry{title}{date}{institution}{location}{description (opt)}
\newcommand{\resentry}[5]{%
    \par\noindent\hspace{1.5em}%
    \begin{minipage}[t]{\dimexpr\textwidth-1.5em\relax}
    \textbf{\textit{#1}} \hfill #2\\
    \textbf{#3}, \textit{#4}\\
    \if\relax\detokenize{#5}\relax\else{\small #5}\\\fi
    \end{minipage}\par
    \vspace{0.5em}
}

\begin{document}

% Header
\begin{center}
    {\Huge\bfseries Sam A. Scivier}\\[0.3em]
    \textit{British \& Canadian Citizen}\\[0.3em]
    Department of Earth Sciences, University of Oxford\\
    South Parks Road, Oxford, OX3 1AN, UK\\[0.8em]
    
    \begin{tabular}{c c c}
        \faEnvelope\ \href{mailto:sam.scivier@earth.ox.ac.uk}{sam.scivier@earth.ox.ac.uk} &
        \faGlobe\ \href{https://sscivier.github.io}{sscivier.github.io} &
        \faLinkedin\ \href{https://www.linkedin.com/in/samscivier/}{samscivier} \\
        \faGithub\ \href{https://github.com/sscivier}{sscivier} &
        \faGraduationCap\ \href{https://scholar.google.com/citations?user=aAvhqzIAAAAJ}{Google Scholar} &
        \faUniversity\ \href{https://www.earth.ox.ac.uk/people/sam-scivier}{Oxford Profile}
    \end{tabular}
\end{center}

\vspace{0.5em}

% Summary
\section*{Summary \& Interests}
\begin{summary}
I am a PhD student in the Department of Earth Sciences at the University of Oxford, developing probabilistic methods for uncertainty quantification in geophysics. 
I hold a Master’s in Physics from the University of Birmingham and have gained industry experience through internships in quantum computing at D-Wave Systems (Canada) and Riverlane (UK). 
My research focuses on Gaussian process-based approaches for probabilistic fusion of geospatial datasets, with applications to earthquake ground motion prediction and seismic hazard assessment. 
I am interested in applying physics-based computational, data science, and machine learning methods to tackle challenges across geoscience, aerospace, sustainability, and emerging technologies, with a particular focus on research opportunities that combine rigorous scientific methodology with practical applications having tangible societal impact.
\end{summary}

% Education - cv-1 compact style
\section*{Education}

\eduentry{Ph.D.}{Geophysics}{University of Oxford (Hertford College)}{Oxford, UK}{10/2022 -- Present}{Expected 2026}{Oxford-NERC DTP in Environmental Research (Full studentship $\sim$ \pounds 120k)}{Paula Koelemeijer, Tarje Nissen-Meyer, Atılım Güneş Baydin}{Gaussian processes for the probabilistic fusion of geophysical datasets -- with application to seismic hazard assessment.}

\eduentry{M.Sci.}{Physics (First Class Honours)}{University of Birmingham}{Birmingham, UK}{10/2018 -- 07/2022}{}{}{Alberto Vecchio}{Graduated as M.Sci. Physics student with highest overall mark. Specialized coursework in theoretical and quantum physics, radar and imaging techniques. \\Final project: Machine learning algorithms for early identification of massive black hole binary mergers for LISA mission (ESA; launch 2035). \\Third year project: Bayesian inference for parameter estimation of binary black hole mergers in LIGO experiment.}

\eduentry{High School Diploma}{British Columbia}{Prince of Wales Secondary School}{Vancouver, Canada}{09/2013 -- 06/2018}{}{}{}{Graduated as highest GPA student. Top scholar for Grades 10, 11, and 12.}

% Research Experience
\section*{Research Experience \& Employment}

\resentry{Ph.D. Researcher}{10/2022 -- Present}{Department of Earth Sciences, University of Oxford}{Oxford, UK}{%
Working on probabilistic methods for uncertainty quantification in geophysics.
Developed Gaussian process-based approach for geospatial data fusion with applications to earthquake ground motion prediction and seismic hazard assessment.
Building collaborations to extend methods to other geophysical problems, and developing open-source software to make methods broadly accessible across the geosciences.
}

\resentry{Quantum Science Intern}{06/2021 -- 08/2021}{Riverlane}{Cambridge, UK}{%
Focused on improving resource efficiency in quantum computation, and developed software for quantum computers using Python.
Collaborated with a multidisciplinary team of physicists, chemists, mathematicians, and software engineers.
Delivered algorithm implementation and research presentation, and co-authored a paper published in the \textit{Journal of Chemical Theory and Computation} (2022).
}

\resentry{Quantum Research Intern}{06/2019 -- 08/2019}{D-Wave Systems}{Burnaby, Canada}{%
Conducted theoretical research in quantum technology and applications, using MATLAB for simulations of nonstoquastic quantum processing and analysis.
Designed optimization protocol for nonstoquastic quantum annealing, and co-authored a paper published in \textit{Physical Review A} (2021).
}

\resentry{Student Science Mentee}{03/2016 -- 04/2016}{DPoint Technologies}{Vancouver, Canada}{%
Worked in a commercial research laboratory, preparing membrane samples and testing them using analytical equipment.
Analyzed results and their implications for commercial applications.
}

% Publications - formatted with reverse numbered references [3], [2], [1]
\section*{Publications}
\renewcommand{\labelenumi}{[\arabic{enumi}]}
\begin{etaremune}[itemsep=4pt, start=3]

\item \textbf{S.A. Scivier}, T. Nissen-Meyer, P. Koelemeijer, and A.G. Baydin, ``Gaussian Processes for Probabilistic Estimates of Earthquake Ground Shaking: A 1-D Proof-of-Concept,'' \textit{arXiv:2412.03299 [physics.geo-ph]} (2024). \href{https://doi.org/10.48550/arXiv.2412.03299}{DOI: 10.48550/arXiv.2412.03299}. Peer-reviewed and presented at ML4PS Workshop at NeurIPS 2024.

\item N.S. Blunt, J. Camps, O. Crawford, R. Izsák, S. Leontica, A. Mirani, A.E. Moylett, \textbf{S.A. Scivier}, et al., ``Perspective on the current state-of-the-art of quantum computing for drug discovery applications,'' \textit{Journal of Chemical Theory and Computation} \textbf{18}, 7001-7023 (2022). \href{https://doi.org/10.1021/acs.jctc.2c00574}{DOI: 10.1021/acs.jctc.2c00574}.

\item E.M. Lykiardopoulou, A. Zucca, \textbf{S.A. Scivier}, and M.H. Amin, ``Improving nonstoquastic quantum annealing with spin-reversal transformations,'' \textit{Physical Review A} \textbf{104}, 012619 (2021). \href{https://doi.org/10.1103/PhysRevA.104.012619}{DOI: 10.1103/PhysRevA.104.012619}.
\end{etaremune}

% Teaching & Outreach - cv-1 compact style
\section*{Teaching \& Outreach}
\cventry{Workshop Leader}{November 2024}{Oxford Intelligent Earth CDT}{}
Gaussian Processes for Probabilistic Earthquake Ground Motion Prediction. Created open-source materials.

% Professional Experience - cv-1 compact style
\section*{Professional Experience}
\cventry{Assistant Programme Coordinator}{June -- August 2020}{Squash British Columbia, Vancouver}{}
COVID-19 pandemic response design. Communications management and virtual event organization.

\cventry{Assistant Squash Professional}{June 2016 -- September 2018}{Jericho Tennis Club, Vancouver}{}
Junior and adult coaching. Positive sports environment development.

% Awards - cv-1 compact style
\section*{Awards \& Recognition}
IAGA/IASPEI Travel Grant (2025) $\bullet$ British Seismology Best Student Poster (2024) $\bullet$ SWJ Smith Prize - Highest M.Sci. Physics Graduate (2022) $\bullet$ Physics Sports Scholarship (2019-2022) $\bullet$ Governor General's Academic Medal (2018) $\bullet$ BC Academic Achievement Scholarship (2018)

% Technical Skills - cv-1 compact style
\section*{Technical Skills}
\textbf{Programming:} Python (7+ years), MATLAB, Bash, HTML \\
\textbf{ML \& Data Science:} TensorFlow, PyTorch, Gaussian Processes, Bayesian inference \\
\textbf{Geophysics:} Finite difference methods, seismic wave propagation, geospatial data \\
\textbf{Tools:} GitHub/GitLab, LaTeX, VSCode \\
\textbf{Methods:} Probabilistic methods, numerical methods, open-source development

% Conferences - cv-1 compact style
\section*{Selected Conferences \& Presentations}
\cventry{Invited Talk}{November 2024}{University of Exeter}{Gaussian process methods for ground motion prediction}

\cventry{Oral Presentation}{September 2025}{IAGA/IASPEI Assembly, Lisbon}{Probabilistic fusion of seismic velocity models}

\cventry{Poster}{December 2024}{NeurIPS ML4PS Workshop, Vancouver}{Gaussian Process workflow for earthquake ground motion}

\cventry{Poster (\textit{Best Student Prize})}{March 2024}{British Seismology Meeting, Reading}{Neural process methods for seismic velocity merging}

\cventry{Workshop}{March 2023}{SPIN ITN, Pitlochry, Scotland}{Earth imaging and inverse problems in geophysics}

% Languages - cv-1 compact style
\section*{Languages}
English (Native) $\bullet$ French (Professional) $\bullet$ Romanian (Basic)

% Extracurricular - cv-1 compact style
\section*{Extracurricular Activities}
\textbf{Competitive Squash:} Silver medal (2019 Canada Winter Games), Gold medal (2022 BUCS Championships), University of Birmingham President (2019-2020) \\
\textbf{Other Interests:} Skiing, hiking, cycling, climbing, photography

% Memberships - cv-1 compact style
\section*{Professional Memberships}
Fellow of the Royal Astronomical Society (2025) $\bullet$ Institute of Physics Member (2018--present)

\end{document}
