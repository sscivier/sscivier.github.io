\documentclass[11pt,a4paper]{article}
\usepackage[utf8]{inputenc}
\usepackage[T1]{fontenc}
\usepackage{lmodern}
\usepackage[margin=0.75in]{geometry}
\usepackage{titlesec}
\usepackage{enumitem}
\usepackage{hyperref}
\usepackage{xcolor}
\usepackage{fontawesome}
\usepackage{microtype}
\usepackage{etaremune}
\usepackage{fancyhdr}

% Color scheme
\definecolor{linkcolor}{RGB}{25,25,112}
% \definecolor{linkcolor}{RGB}{14,165,233}

% Hyperref setup
\hypersetup{
    colorlinks=true,
    linkcolor=linkcolor,
    urlcolor=linkcolor,
    citecolor=linkcolor,
    pdfauthor={Sam A. Scivier},
    pdftitle={Sam A. Scivier - Curriculum Vitae},
    pdfsubject={Curriculum Vitae},
    pdfkeywords={geophysics, machine learning, uncertainty quantification, earthquake prediction}
}

% Section formatting 
\titleformat{\section}
{\large\bfseries}
{}{0pt}{\MakeUppercase}

\titleformat{\subsection}
{\normalsize\bfseries}
{}{0pt}{}

% Professional summary formatting - narrower margins and slightly smaller font
\newenvironment{summary}
  {\begin{quote}\small}
  {\end{quote}}

% Compact list formatting
\setlist[itemize]{leftmargin=1.5em, topsep=0pt, itemsep=2pt}
\setlist[enumerate]{leftmargin=1.5em, topsep=0pt, itemsep=2pt}

% Page style with headers
\pagestyle{fancy}
\fancyhf{} % clear all header and footer fields
\fancyhead[L]{Sam A. Scivier}
\fancyhead[R]{Page \thepage\ of \pageref{LastPage}}
\renewcommand{\headrulewidth}{0.4pt}
\setlength{\headheight}{14.5pt}

% Compact entry command
\newcommand{\cventry}[4]{%
    \noindent\textbf{#1} \hfill #2\\
    \textit{#3} \hfill #4\\[0.3em]
}
% Education entry command
% Usage: \eduentry{degree}{subject}{institution}{location}{date}{expected date (opt)}{funding (opt)}{supervisors (opt)}{description (opt)}
\newcommand{\eduentry}[9]{%
    \par\noindent\hspace{1.5em}%
    \begin{minipage}[t]{\dimexpr\textwidth-1.5em\relax}
    \textbf{\textit{#1}}, \textbf{#2} \hfill #5\\
    \textbf{#3}, \textit{#4} \hfill #6\\
    \if\relax\detokenize{#7}\relax\else {\small Funding: #7}\\\fi
    \if\relax\detokenize{#8}\relax\else {\small Advisor(s): #8}\\\fi
    \if\relax\detokenize{#9}\relax\else{\small #9}\\\fi
    \end{minipage}\par
    \vspace{0.5em}
}

% Teaching entry command
% Usage: \teachentry{role}{date}{course title}{institution}{location}{description (opt)}
\newcommand{\teachentry}[6]{%
    \par\noindent\hspace{1.5em}%
    \begin{minipage}[t]{\dimexpr\textwidth-1.5em\relax}
    \textbf{\textit{#1}} \hfill #2\\
    \textbf{#3}\\
    \textbf{#4}, \textit{#5}\\
    \if\relax\detokenize{#6}\relax\else{\small #6}\\\fi
    \end{minipage}\par
    \vspace{0.5em}
}

% Research entry command
% Usage: \resentry{title}{date}{institution}{location}{description (opt)}
\newcommand{\resentry}[5]{%
    \par\noindent\hspace{1.5em}%
    \begin{minipage}[t]{\dimexpr\textwidth-1.5em\relax}
    \textbf{\textit{#1}} \hfill #2\\
    \textbf{#3}, \textit{#4}\\
    \if\relax\detokenize{#5}\relax\else{\small #5}\\\fi
    \end{minipage}\par
    \vspace{0.5em}
}

% Award entry command
% Usage: \awentry{award}{issued by}{date}{description (opt)}
\newcommand{\awentry}[4]{%
    \par\noindent\hspace{1.5em}%
    \begin{minipage}[t]{\dimexpr\textwidth-1.5em\relax}
    \textbf{\textit{#1}} \hfill #3\\
    \textbf{#2} \\
    \if\relax\detokenize{#4}\relax\else{\small #4}\\\fi
    \end{minipage}\par
    \vspace{0.5em}
}

% Talk entry command
% Usage: \talkentry{title}{date}{event}{location}{description (opt)}
\newcommand{\talkentry}[5]{%
    \par\noindent\hspace{1.5em}%
    \begin{minipage}[t]{\dimexpr\textwidth-1.5em\relax}
    \textbf{\textit{#1}} \hfill #2\\
    \textbf{#3}, \textit{#4}\\
    \if\relax\detokenize{#5}\relax\else{\small #5}\\\fi
    \end{minipage}\par
    \vspace{0.5em}
}

% Professional development entry command
% Usage: \profdeventry{title}{date}{organizer}{location}{description (opt)}
\newcommand{\profdeventry}[5]{%
    \par\noindent\hspace{1.5em}%
    \begin{minipage}[t]{\dimexpr\textwidth-1.5em\relax}
    \textbf{\textit{#1}} \hfill #2\\
    \textbf{#3}, \textit{#4}\\
    \if\relax\detokenize{#5}\relax\else{\small #5}\\\fi
    \end{minipage}\par
    \vspace{0.5em}
}

\begin{document}

% Suppress header on first page
\thispagestyle{empty}

% Header
\begin{center}
    {\Huge\bfseries Sam A. Scivier}\\[0.3em]
    \textit{British \& Canadian Citizen}\\[0.3em]
    Department of Earth Sciences, University of Oxford\\
    South Parks Road, Oxford, OX3 1AN, UK\\[0.8em]
    
    \begin{tabular}{c c c}
        \faEnvelope\ \href{mailto:sam.scivier@earth.ox.ac.uk}{sam.scivier@earth.ox.ac.uk} &
        \faGlobe\ \href{https://sscivier.github.io}{sscivier.github.io} &
        \faLinkedin\ \href{https://www.linkedin.com/in/samscivier/}{samscivier} \\
        \faGithub\ \href{https://github.com/sscivier}{sscivier} &
        \faGraduationCap\ \href{https://scholar.google.com/citations?user=aAvhqzIAAAAJ}{Google Scholar} &
        \faUniversity\ \href{https://www.earth.ox.ac.uk/people/sam-scivier}{Oxford Profile}
    \end{tabular}
\end{center}

\vspace{0.5em}

% Summary
\section*{Summary \& Interests}
\begin{summary}
I am a PhD student in the Department of Earth Sciences at the University of Oxford, developing probabilistic methods for uncertainty quantification in geophysics. 
I hold a Master’s in Physics from the University of Birmingham and have gained industry experience through internships in quantum computing at D-Wave Systems (Canada) and Riverlane (UK). 
My research focuses on Gaussian process-based approaches for probabilistic fusion of geospatial datasets, with applications to earthquake ground motion prediction and seismic hazard assessment. 
I am interested in applying physics-based computational, data science, and machine learning methods to tackle challenges across geoscience, aerospace, sustainability, and emerging technologies, with a particular focus on research opportunities that combine rigorous scientific methodology with practical applications having tangible societal impact.
\end{summary}

% Education - cv-1 compact style
\section*{Education}

\eduentry{Ph.D.}{Geophysics}{University of Oxford (Hertford College)}{Oxford, UK}{10/2022 -- Present}{Expected 2026}{Oxford-NERC DTP in Environmental Research (Full studentship $\sim$ \pounds 120k)}{Paula Koelemeijer, Tarje Nissen-Meyer, Atılım Güneş Baydin}{Gaussian processes for the probabilistic fusion of geophysical datasets -- with application to seismic hazard assessment.}

\eduentry{M.Sci.}{Physics (First Class Honours)}{University of Birmingham}{Birmingham, UK}{10/2018 -- 07/2022}{}{}{Alberto Vecchio}{Graduated as M.Sci. Physics student with highest overall mark. Specialized coursework in theoretical and quantum physics, radar and imaging techniques. \\Final project: Machine learning algorithms for early identification of massive black hole binary mergers for LISA mission (ESA; launch 2035). \\Third year project: Bayesian inference for parameter estimation of binary black hole mergers in LIGO experiment.}

\eduentry{High School Diploma}{British Columbia}{Prince of Wales Secondary School}{Vancouver, Canada}{09/2013 -- 06/2018}{}{}{}{Graduated as highest GPA student. Top scholar for Grades 10, 11, and 12.}

% Research Experience
\section*{Research Experience \& Employment}

\resentry{Ph.D. Researcher}{10/2022 -- Present}{Department of Earth Sciences, University of Oxford}{Oxford, UK}{%
Working on probabilistic methods for uncertainty quantification in geophysics.
Developed Gaussian process-based approach for geospatial data fusion with applications to earthquake ground motion prediction and seismic hazard assessment.
Building collaborations to extend methods to other geophysical problems, and developing open-source software to make methods broadly accessible across the geosciences.
}

\resentry{Quantum Science Intern}{06/2021 -- 08/2021}{Riverlane}{Cambridge, UK}{%
Focused on improving resource efficiency in quantum computation, and developed software for quantum computers using Python.
Collaborated with a multidisciplinary team of physicists, chemists, mathematicians, and software engineers.
Delivered algorithm implementation and research presentation, and co-authored a paper published in the \textit{Journal of Chemical Theory and Computation} (2022).
}

\resentry{Quantum Research Intern}{06/2019 -- 08/2019}{D-Wave Systems}{Burnaby, Canada}{%
Conducted theoretical research in quantum technology and applications, using MATLAB for simulations of nonstoquastic quantum processing and analysis.
Designed optimization protocol for nonstoquastic quantum annealing, and co-authored a paper published in \textit{Physical Review A} (2021).
}

\resentry{Student Science Mentee}{03/2016 -- 04/2016}{DPoint Technologies}{Vancouver, Canada}{%
Worked in a commercial research laboratory, preparing membrane samples and testing them using analytical equipment.
Analyzed results and their implications for commercial applications.
}

% Publications - formatted with reverse numbered references [3], [2], [1]
\section*{Publications}
\renewcommand{\labelenumi}{[\arabic{enumi}]}
\begin{etaremune}[itemsep=4pt, start=3]

\item \textbf{S.A. Scivier}, T. Nissen-Meyer, P. Koelemeijer, and A.G. Baydin, ``Gaussian Processes for Probabilistic Estimates of Earthquake Ground Shaking: A 1-D Proof-of-Concept,'' \textit{arXiv:2412.03299 [physics.geo-ph]} (2024). \href{https://doi.org/10.48550/arXiv.2412.03299}{DOI: 10.48550/arXiv.2412.03299}. Peer-reviewed and presented at ML4PS Workshop at NeurIPS 2024.

\item N.S. Blunt, J. Camps, O. Crawford, R. Izsák, S. Leontica, A. Mirani, A.E. Moylett, \textbf{S.A. Scivier}, et al., ``Perspective on the current state-of-the-art of quantum computing for drug discovery applications,'' \textit{Journal of Chemical Theory and Computation} \textbf{18}, 7001-7023 (2022). \href{https://doi.org/10.1021/acs.jctc.2c00574}{DOI: 10.1021/acs.jctc.2c00574}.

\item E.M. Lykiardopoulou, A. Zucca, \textbf{S.A. Scivier}, and M.H. Amin, ``Improving nonstoquastic quantum annealing with spin-reversal transformations,'' \textit{Physical Review A} \textbf{104}, 012619 (2021). \href{https://doi.org/10.1103/PhysRevA.104.012619}{DOI: 10.1103/PhysRevA.104.012619}.
\end{etaremune}

\section*{Conference Presentations}
\renewcommand{\labelenumi}{[\arabic{enumi}]}
\begin{etaremune}[itemsep=4pt, start=4]

\item \textbf{S.A. Scivier}, P. Koelemeijer, T. Nissen-Meyer, and A.M. Mag, ``Probabilistic fusion of seismic velocity models using Gaussian processes,'' \textit{Oral presentation}, IAGA/IASPEI Joint Scientific Meeting, Lisbon, Portugal (September 2025). Session J04 - Data assimilation and Machine Learning: Challenges and Leveraging New Opportunities.

\item \textbf{S.A. Scivier}, T. Nissen-Meyer, P. Koelemeijer, and A.G. Baydin, ``Gaussian Processes for Probabilistic Estimates of Earthquake Ground Shaking: A 1-D Proof-of-Concept,'' \textit{Poster presentation}, ML4PS Workshop at NeurIPS 2024, Vancouver, Canada (December 2024).

\item \textbf{S.A. Scivier}, T. Nissen-Meyer, P. Koelemeijer, and A.G. Baydin, ``Physics-based probabilistic estimates of earthquake ground shaking: A synthetic 1D proof of concept,'' \textit{Poster presentation}, NERC DTP Student Conference 2024, Oxford, UK (June 2024).

\item \textbf{S.A. Scivier}, T. Nissen-Meyer, P. Koelemeijer, and A.G. Baydin, ``Probabilistic merging of seismic velocity datasets using deep learning: a case study on synthetic data,'' \textit{Poster presentation}, British Seismology Meeting 2024, Reading, UK (March 2024). \textbf{Winner of Best Student Poster Prize}.

\end{etaremune}

\section*{Software \& Repositories}
\renewcommand{\labelenumi}{[\arabic{enumi}]}
\begin{etaremune}[itemsep=4pt, start=2]

\item \textbf{S.A. Scivier}, T. Nissen-Meyer, P. Koelemeijer, and A.G. Baydin, ``Gaussian Processes for Probabilistic Estimates of Earthquake Ground Shaking: A 1-D Proof-of-Concept,'' \textit{Software} (2024). Zenodo. \href{https://doi.org/10.5281/zenodo.14545465}{DOI: 10.5281/zenodo.14545465}. GitHub: \href{https://github.com/sscivier/gp-prob-earthquake-shaking}{github.com/sscivier/gp-prob-earthquake-shaking}.

\item \textbf{S.A. Scivier}, ``Workshop materials on Gaussian Processes for probabilistic earthquake ground motion prediction,'' \textit{Educational Software Repository} (2024). GitHub: \href{https://github.com/sscivier/intelligent-earth-cdt-earthquakes-gp}{github.com/sscivier/intelligent-earth-cdt-earthquakes-gp}. Interactive Jupyter notebook with educational examples and comprehensive utility modules for GP modeling, wave propagation, and visualization.

\end{etaremune}

\section*{Invited Talks}
\talkentry{Towards physics-based probabilistic estimates of earthquake ground motion using Gaussian processes}{11/2024}{Mathematics and Statistics Seminar Series, University of Exeter}{Exeter, UK}{Seminar on developing probabilistic methods for incorporating uncertainties from seismic velocity model inconsistencies in earthquake ground motion prediction using Gaussian process regression.}

% Teaching & Outreach
\section*{Teaching \& Outreach}

\teachentry{Workshop Leader}{11/2024}{Gaussian Processes for Probabilistic Earthquake Ground Motion Prediction}{Oxford Intelligent Earth CDT, University of Oxford}{Oxford, UK}{%
Led workshop for first-year PhD students on probabilistic fusion of seismic velocity models using Gaussian Processes, with application to seismic hazard.
Created open-source Jupyter notebook with interactive examples demonstrating data fusion and uncertainty quantification. 
Designed progressive exercises covering engineering safety assessment and computational optimization. 
Materials available at: \href{https://github.com/sscivier/intelligent-earth-cdt-earthquakes-gp}{github.com/sscivier/intelligent-earth-cdt-earthquakes-gp}.
}

% Non-Technical Professional Experience
\section*{Non-Technical Professional Experience}

\resentry{Assistant Programme Coordinator}{06/2020 -- 08/2020}{Squash British Columbia}{Vancouver, Canada}{%
Worked with the Executive Director to design, coordinate, and communicate Squash BC’s response to the COVID-19 pandemic. 
Organized and executed a panel discussion on university opportunities and experiences for an audience of competitive junior squash players.
}

\resentry{Assistant Squash Professional}{06/2016 -- 09/2018}{Jericho Tennis Club}{Vancouver, Canada}{%
Worked with junior and adult players to improve their squash game through tailored (private and group) coaching sessions.
Created a positive and inclusive sports environment, and served as a role model for junior squash players.
}

% Awards 
\section*{Awards \& Recognition}
\awentry{IAGA/IASPEI Travel Grant}{IAGA/IASPEI Joint Scientific Meeting}{09/2025}{Waived registration fee to attend the IAGA/IASPEI Joint Scientific Meeting in Lisbon, Portugal, for presenting my work on probabilistic fusion of seismic velocity models using Gaussian Processes. (Value €290)}
\awentry{Best Student Poster -- British Seismology Meeting}{International Seismological Centre}{03/2024}{Awarded for my poster presentation at the British Seismology Meeting in Reading, UK. (Value \pounds100)}
\awentry{SWJ Smith Prize}{School of Physics and Astronomy, University of Birmingham}{07/2022}{Awarded to the M.Sci. Physics student graduating with the highest overall mark. (Value \pounds500)}
\awentry{Physics Sports Bursary}{School of Physics and Astronomy, University of Birmingham}{2019 -- 2022}{Awarded for combined academic and athletic achievement. (Value \pounds800/year)}
\awentry{Academic Achievement Scholarship}{School of Physics and Astronomy, University of Birmingham}{01/2019}{}
\awentry{Governor General's Academic Medal}{Government of Canada}{06/2018}{The most prestigious award that students in Canadian schools can receive. Awarded to the student graduating with the highest GPA from a high school.}

% Technical Skills 
\section*{Technical Skills}
\par\noindent\hspace{1.5em}%
\begin{minipage}[t]{\dimexpr\textwidth-1.5em\relax}
\textbf{Programming:} Python (7+ years), Git, MATLAB, Bash, HTML \\
\textbf{ML \& Data Science:} TensorFlow, PyTorch, Gaussian Processes, Bayesian inference \\
\textbf{Geophysics:} Finite difference methods, seismic wave propagation, geospatial data \\
\textbf{Tools:} GitHub/GitLab, LaTeX, VSCode \\
\textbf{Methods:} Probabilistic methods, machine learning, numerical methods, open-source development
\end{minipage}\par
\vspace{0.5em}

\section*{Professional Development}

\profdeventry{SPIN Short Course 3: ``Interrogating the Restless Earth''}{03/2023}{SPIN ITN}{Pitlochry, Scotland, UK}{5-day intensive workshop on Earth imaging, monitoring and inverse problems in geophysics, covering mathematical theory and practical applications including optimal experimental design, variational inference, uncertainties, transdimensional inference, and machine learning methods.}

% Languages - cv-1 compact style
\section*{Languages}
\par\noindent\hspace{1.5em}%
\begin{minipage}[t]{\dimexpr\textwidth-1.5em\relax}
English (Native) $\bullet$ French (Professional) $\bullet$ Romanian (Basic)
\end{minipage}\par
\vspace{0.5em}

% Extracurricular 
\section*{Extracurricular Activities}
\par\noindent\hspace{1.5em}%
\begin{minipage}[t]{\dimexpr\textwidth-1.5em\relax}
\textbf{Competitive Squash:} Gold medal (2022 BUCS Team Championships); Silver Medal (2019 BUCS Team Championships); Silver medal (2019 Canada Winter Games); University of Birmingham Squash Club -- President (2019-2020), Media Secretary (2020-2021), Welfare Secretary (2021-2022) \\

\noindent\textbf{Other Interests:} Skiing, hiking, cycling, climbing, photography
\end{minipage}\par
\vspace{0.5em}

% Memberships 
\section*{Professional Memberships}
\par\noindent\hspace{1.5em}%
\begin{minipage}[t]{\dimexpr\textwidth-1.5em\relax}
Fellow of the Royal Astronomical Society (Elected 02/2025) $\bullet$ Institute of Physics Member (2018--Present)
\end{minipage}\par
\vspace{0.5em}

% Add last updated date at bottom right
\vfill
\begin{flushright}
\small\textit{Last updated: \today}
\end{flushright}

\label{LastPage}
\end{document}
