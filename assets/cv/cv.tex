\documentclass[11pt,a4paper]{article}
\usepackage[utf8]{inputenc}
\usepackage[T1]{fontenc}
\usepackage{lmodern}
\usepackage[margin=0.75in]{geometry}
\usepackage{titlesec}
\usepackage{enumitem}
\usepackage{hyperref}
\usepackage{xcolor}
\usepackage{fontawesome}
\usepackage{microtype}

% Color scheme
\definecolor{darkblue}{RGB}{25,25,112}
\definecolor{linkcolor}{RGB}{14,165,233}

% Hyperref setup
\hypersetup{
    colorlinks=true,
    linkcolor=linkcolor,
    urlcolor=linkcolor,
    citecolor=linkcolor,
    pdfauthor={Sam A. Scivier},
    pdftitle={Sam A. Scivier - Curriculum Vitae},
    pdfsubject={Curriculum Vitae},
    pdfkeywords={geophysics, machine learning, uncertainty quantification, earthquake prediction}
}

% Custom section formatting
\titleformat{\section}
{\Large\bfseries\color{darkblue}}
{\thesection}{0em}{}[\titlerule]

\titleformat{\subsection}
{\large\bfseries}
{\thesubsection}{0em}{}

% Custom list formatting
\setlist[itemize]{leftmargin=*,nosep,before=\vspace{-0.5\baselineskip},after=\vspace{0.5\baselineskip}}
\setlist[enumerate]{leftmargin=*,nosep}

% Fix spacing for cventry items
\setlength{\parindent}{0pt}
\setlength{\parskip}{0pt}

% Remove page numbers
\pagestyle{empty}

% Custom commands
\newcommand{\cventry}[4]{%
    \noindent\textbf{#1} \hfill #2\\
    \textit{#3} \hfill #4\\[0.1em]
}

\newcommand{\publication}[1]{%
    \item #1
}

\begin{document}

% Header
\begin{center}
    {\Huge\bfseries Sam A. Scivier}\\[0.3em]
    \textit{British \& Canadian Citizen}\\[0.3em]
    Department of Earth Sciences, University of Oxford\\
    South Parks Road, Oxford, OX3 1AN, UK\\[0.8em]
    
    \begin{tabular}{c c c}
        \faEnvelope\ \href{mailto:sam.scivier@earth.ox.ac.uk}{sam.scivier@earth.ox.ac.uk} &
        \faGlobe\ \href{https://sscivier.github.io}{sscivier.github.io} &
        \faLinkedin\ \href{https://www.linkedin.com/in/samscivier/}{samscivier} \\
        \faGithub\ \href{https://github.com/sscivier}{sscivier} &
        \faGraduationCap\ \href{https://scholar.google.com/citations?user=aAvhqzIAAAAJ}{Google Scholar} &
        \faUniversity\ \href{https://www.earth.ox.ac.uk/people/sam-scivier}{Oxford Profile}
    \end{tabular}
\end{center}

\vspace{0.5em}

% Professional Summary
\section*{Professional Summary}

I am a PhD student in the Department of Earth Sciences at the University of Oxford, developing probabilistic methods for uncertainty quantification in geophysics. I hold a Master's in Physics from the University of Birmingham and have gained industry experience through internships in quantum computing at D-Wave Systems (Canada) and Riverlane (UK). My research focuses on Gaussian process-based approaches for probabilistic fusion of geospatial datasets, with applications to earthquake ground motion prediction and seismic hazard assessment. I am interested in applying physics-based computational, data science, and machine learning methods to tackle challenges across geoscience, aerospace, sustainability, and emerging technologies, with a particular focus on research opportunities that combine rigorous scientific methodology with practical applications having tangible societal impact.

% Education
\section*{Education}

\cventry{PhD in Earth Sciences}{October 2022 -- present}{University of Oxford, England}{Expected 2026}
\begin{itemize}
    \item Funding: Oxford-NERC DTP in Environmental Research (Full studentship $\sim$£120k)
    \item Research: Developing probabilistic methods for uncertainty quantification in physics-based seismic hazard assessment
    \item Focus: Gaussian process-based approaches for probabilistic fusion of overlapping geospatial datasets
\end{itemize}

\cventry{M.Sci. Physics (First Class Honours)}{October 2018 -- July 2022}{University of Birmingham, England}{}
\begin{itemize}
    \item Specialized coursework in theoretical and quantum physics, radar and imaging techniques
    \item Final year project: Machine learning algorithms for early identification of massive black hole binary mergers for LISA mission (ESA; launch 2035)
    \item Third year project: Bayesian inference for parameter estimation of binary black hole mergers
\end{itemize}

\cventry{British Columbia High School Diploma}{September 2013 -- June 2018}{Prince of Wales Secondary School, Vancouver, Canada}{}
\begin{itemize}
    \item Graduated as highest GPA student with A grades in all Grade 12 subjects
    \item Top scholar for Grades 10, 11, and 12
\end{itemize}

% Research Experience
\section*{Research Experience}

\cventry{PhD Researcher}{October 2022 -- present}{Department of Earth Sciences, University of Oxford}{}
\begin{itemize}
    \item Working on probabilistic methods for uncertainty quantification in physics-based seismic hazard assessment
    \item Developed Gaussian process-based approaches for probabilistic fusion of overlapping geospatial datasets
    \item Building collaborations to extend methods to other geophysical problems
    \item Developing open-source software to make methods broadly accessible across geosciences
\end{itemize}

\cventry{Quantum Science Intern}{June -- August 2021}{Riverlane, Cambridge, UK}{}
\begin{itemize}
    \item Focused on improving resource efficiency in quantum computation
    \item Developed software for quantum computers using Python
    \item Worked in multidisciplinary team of physicists, chemists, mathematicians, and software engineers
    \item Delivered algorithm implementation and research presentation
\end{itemize}

\cventry{Quantum Research Intern}{June -- August 2019}{D-Wave Systems Inc., Burnaby, Canada}{}
\begin{itemize}
    \item Conducted theoretical research in quantum technology and applications
    \item Used MATLAB for simulations of nonstoquastic quantum processing and analysis
    \item Designed optimization protocol for nonstoquastic quantum annealing
    \item Co-authored paper published in Physical Review A (2021)
\end{itemize}

\cventry{Student Science Mentorship Programme Researcher}{March -- April 2016}{DPoint Technologies Inc., Vancouver, Canada}{}
\begin{itemize}
    \item Worked part-time in commercial research laboratory during high school
    \item Prepared membrane samples and conducted testing using analytical equipment
    \item Assessed results for commercial energy recovery ventilation applications
\end{itemize}

% Publications
\section*{Publications}

\begin{enumerate}
\publication{\textbf{S.A. Scivier}, T. Nissen-Meyer, P. Koelemeijer, and A.G. Baydin, ``Gaussian Processes for Probabilistic Estimates of Earthquake Ground Shaking: A 1-D Proof-of-Concept,'' \textit{arXiv:2412.03299 [physics.geo-ph]} (2024). \href{https://doi.org/10.48550/arXiv.2412.03299}{DOI: 10.48550/arXiv.2412.03299}. Peer-reviewed and presented at ML4PS Workshop at NeurIPS 2024.}

\publication{N.S. Blunt, J. Camps, O. Crawford, R. Izsák, S. Leontica, A. Mirani, A.E. Moylett, \textbf{S.A. Scivier}, C. Sunderhauf, P. Schopf, et al., ``Perspective on the current state-of-the-art of quantum computing for drug discovery applications,'' \textit{Journal of Chemical Theory and Computation} \textbf{18}, 7001-7023 (2022). \href{https://doi.org/10.1021/acs.jctc.2c00574}{DOI: 10.1021/acs.jctc.2c00574}}

\publication{E.M. Lykiardopoulou, A. Zucca, \textbf{S.A. Scivier}, and M.H. Amin, ``Improving nonstoquastic quantum annealing with spin-reversal transformations,'' \textit{Physical Review A} \textbf{104}, 012619 (2021). \href{https://doi.org/10.1103/PhysRevA.104.012619}{DOI: 10.1103/PhysRevA.104.012619}}
\end{enumerate}

% Teaching and Outreach
\section*{Teaching \& Outreach}

\cventry{Gaussian Processes for Probabilistic Earthquake Ground Motion Prediction}{November 2024}{Workshop Leader, Oxford Intelligent Earth CDT}{}
\begin{itemize}
    \item Led workshop for first-year PhD students on probabilistic fusion of seismic velocity models using Gaussian Processes
    \item Created open-source Jupyter notebook with interactive examples demonstrating data fusion and uncertainty quantification
    \item Designed progressive exercises covering engineering safety assessment and computational optimization
    \item Materials available at: \href{https://github.com/sscivier/intelligent-earth-cdt-earthquakes-gp}{github.com/sscivier/intelligent-earth-cdt-earthquakes-gp}
\end{itemize}

% Work Experience
\section*{Non-Technical Professional Experience}

\cventry{Assistant Programme Coordinator}{June -- August 2020}{Squash British Columbia, Vancouver, Canada}{}
\begin{itemize}
    \item Collaborated with Executive Director to design Squash BC's COVID-19 pandemic response
    \item Managed communications to member facilities through website, newsletters, and online meetings
    \item Organized virtual panel discussion on university opportunities for competitive junior squash players
\end{itemize}

\cventry{Part-Time Assistant Squash Professional}{June 2016 -- September 2018}{Jericho Tennis Club, Vancouver, Canada}{}
\begin{itemize}
    \item Coached junior and adult squash players in private and group lessons
    \item Created positive sports environment and served as role model for junior squash players
\end{itemize}

% Awards and Recognition
\section*{Awards \& Recognition}

\begin{itemize}
    \item IAGA/IASPEI 2025 Travel Grant -- Free registration (€290 value) (2025)
    \item British Seismology Meeting 2024 -- Best Student Poster Prize (2024)
    \item University of Birmingham School of Physics and Astronomy SWJ Smith Prize -- M.Sci. Physics graduate with highest overall mark (2022)
    \item University of Birmingham Physics Sports Scholarship (2019-2022) -- Combined academic and athletic excellence
    \item Canadian Governor General's Academic Medal -- Highest GPA graduate (2018)
    \item British Columbia Academic Achievement Scholarship (2018)
    \item University of Birmingham School of Physics and Astronomy Academic Achievement Scholarship (2018/19)
    \item SFU Applied Sciences Math 11 Award, Simon Fraser University (2017)
\end{itemize}

% Technical Skills
\section*{Technical Skills}

\textbf{Programming:} Python (7+ years), MATLAB, Bash, HTML\\
\textbf{Machine Learning \& Data Science:} TensorFlow, PyTorch, Weights \& Biases, Gaussian Processes, Bayesian inference\\
\textbf{Geophysical Modeling:} Finite difference methods, seismic wave propagation, geospatial data handling\\
\textbf{Software \& Tools:} GitHub/GitLab, LaTeX, Overleaf, VSCode, Microsoft Office\\
\textbf{Research Methods:} Probabilistic methods, statistical analysis, machine learning, numerical methods, scientific computing, open-source development

% Conferences and Presentations
\section*{Conferences \& Presentations}

\subsection*{Invited Talks}
\cventry{Towards physics-based probabilistic estimates of earthquake ground motion using Gaussian processes}{November 13, 2024}{Department of Mathematics and Statistics, University of Exeter}{}
\begin{itemize}
    \item Presented methods for incorporating uncertainties from seismic velocity model inconsistencies in earthquake ground motion prediction using Gaussian process regression
\end{itemize}

\subsection*{Conference Presentations}
\cventry{Probabilistic fusion of seismic velocity models using Gaussian processes}{September 4, 2025}{IAGA/IASPEI Joint Scientific Assembly, Lisbon, Portugal}{Oral presentation}
\begin{itemize}
    \item Session J04b: Data assimilation and Machine Learning
    \item Presented scalable Gaussian process methods for probabilistic fusion of overlapping seismic velocity models to improve uncertainty quantification in physics-based seismic hazard analysis
\end{itemize}

\cventry{Gaussian Processes for Probabilistic Estimates of Earthquake Ground Shaking: A 1-D Proof-of-Concept}{December 2024}{ML4PS Workshop at NeurIPS 2024, Vancouver, Canada}{Poster presentation}
\begin{itemize}
    \item Demonstrated proof-of-concept workflow for incorporating uncertainties from seismic velocity model inconsistencies into ground motion predictions using Gaussian processes
\end{itemize}

\cventry{Physics-based probabilistic estimates of earthquake ground shaking: A synthetic 1D proof of concept}{June 2024}{NERC DTP Student Conference, Oxford, UK}{Poster presentation}
\begin{itemize}
    \item Presented initial results on probabilistic approaches for earthquake ground motion prediction accounting for velocity model uncertainties
\end{itemize}

\cventry{Probabilistic merging of seismic velocity datasets using deep learning: A case study on synthetic data}{March 2024}{British Seismology Meeting, Reading, UK}{Poster presentation -- \textit{Best Student Poster Prize}}
\begin{itemize}
    \item Demonstrated neural process-based methods for merging overlapping seismic velocity datasets and their impact on earthquake ground motion simulations
\end{itemize}

\subsection*{Professional Development}
\cventry{Interrogating the Restless Earth}{March 19--24, 2023}{SPIN ITN (Seismology and Plate tectonics Imaging Network), Pitlochry, Scotland}{5-day intensive workshop}
\begin{itemize}
    \item Earth imaging, monitoring and inverse problems in geophysics
    \item Topics: optimal experimental design, variational inference, uncertainties, transdimensional inference, and machine learning methods
\end{itemize}

% Languages
\section*{Languages}

\textbf{English:} Native proficiency \quad \textbf{French:} Professional working proficiency \quad \textbf{Romanian:} Basic proficiency

% Extracurricular Activities
\section*{Extracurricular Activities}

\subsection*{Competitive Squash}
\begin{itemize}
    \item Silver medal, 2019 Canada Winter Games; Gold medal, 2022 BUCS Squash Team Championships
    \item 1st team player, University of Birmingham (2018-2022)
    \item President, University of Birmingham Squash Club (2019-2020)
    \item Sport Colours Award for outstanding contribution (2020)
\end{itemize}

\subsection*{Other Interests}
Skiing, hiking, road/mountain biking, bouldering/climbing, photography

% Professional Memberships
\section*{Professional Memberships}

\begin{itemize}
    \item Fellow of the Royal Astronomical Society (elected February 14, 2025)
    \item Institute of Physics Member (Studying) (2018 -- present)
\end{itemize}

\end{document}
